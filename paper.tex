\documentclass[11pt]{sty/oecu-thesis}
\usepackage{cite}
\usepackage[dvipdfmx]{graphicx}
\usepackage{subfigure}
\usepackage{sty/lcaption}
\usepackage{times}
\usepackage{url}
\usepackage{amsmath}

% もう少し具体的なタイトルにする。
\title[優先順位の設定とタイマーが付いたTODOリストの開発]{優先順位の設定とタイマーが付いたTODOリストの開発}
\author{村岡 尚輝}
\date{{令和}\rensuji{2}年\rensuji{12}月\rensuji{14}日}
\学生番号{HT18A098}
\指導教員{久松 潤之 准教授}

% 特別研究の場合はコメントをはずす.卒業研究の場合はコメントアウトする.
%\論文種別{特別研究論文}
\年度{令和2}

\所属{総合情報学部 情報学科}


\begin{document}


\newcommand{\insertfigeps}[2]{%
  \begin{figure}[tb]
    \begin{center}
      \leavevmode
      \includegraphics[width=.80\textwidth]%
      {figure/#1.eps}
      \lcaption{#2}
      \label{fig:#1}
    \end{center}
  \end{figure}}

\newcommand{\insertfigpng}[2]{%
  \begin{figure}[tb]
    \begin{center}
      \leavevmode
      \includegraphics[width=.80\textwidth]%
      {figure/#1.png}
      \lcaption{#2}
      \label{fig:#1}
    \end{center}
  \end{figure}}



%\makeextratitle
\maketitle
\pagenumbering{roman}
\begin{abstract}
  アブストアブストアブストアブストアブストアブストアブストアブストアブスト
  
\keywords % 主な用語
  キーワード1\quad
  キーワード2\quad
  キーワード3\quad
\end{abstract}


\tableofcontents
% 以下の二つは,論文のフォーマットにそっていないが,
% 確認用のためにつける.論文提出時には,コメントアウトする.
\listoffigures
\listoftables
\cleardoublepage

\setcounter{page}{1}
\pagenumbering{arabic}

\chapter{はじめに}
\label{cha:intro}

従来のTODOリストを使う目的はやるべきこと(タスク)をリストアップし、情報を整理することで時間を使い方を改善する。
複数の作業が重なったとき、何から進めればいいか混乱してしまい非効率になる可能性を防ぐことができる。
また、タスクを一つ一つこなしリストから消していくことでじぶんはこれだけやったんだと満足感が得られモチベーションの向上にもつながる。
情報を可視化し明確にすることで生産性を高めることを目的としている。さらに、自分のやるべきことと期限をメモ帳やカレンダーなどに書き込みそれを実行する習慣をつけることで
効率的な時間管理術を身につけることができる。

しかし、実際には時間のかかるタスクを残してしまいすぐ終わるものばかりをこなしてしまったり、リストに書くだけ書いてやらなくなる人が多い。
やるべきことをリストアップしてるだけでどれからやればいいかの優先順位がないので効率が悪くなります。どのくらいの時間で終わるかがわからないので優先順位をつける時も混乱してしまう。
先ほども言ったように時間のかかるような難しいタスクをのしがちになるので取り組んだが途中で投げ出してしまうというケースも少なくはない。

そこで、本研究では優先順位がなく非効率になることを改善する為、優先順位機能の開発をする。これによりどれから取り組めばいいかを可視化させ効率化を図る。

次に予想終了時間を設定できるようにする。優先順位機能を開発して優先順位をつけるといっても具体的な基準、目途がないとただ優先順位をつけるだけではタスクをこなすモチベーションにはつながらない。
予想終了時間を設定できるようにすることでよりはっきりと明瞭にし、情報の整理の手助けをすることでモチベーションの口上を図る。

次にタイマー機能の開発をする。取り組む時間を設定して、集中力をあげることを目的とする。時間を決めずにだらだらとタスクを行っても良いことはない。
取り組む時間を決めることで「その時間まで頑張ろう」という気持ちになり、無駄な時間を使わず少ない時間でタスクを終わらせられることを目標とする。
また、時間設定を繰り返し行うことでそれらにかかる時間をある程度予測できるようになり、より精密な時間管理を行うことができる。
これら3つの機能を追加することで、タスクをリストに書くだけでやらなくなってしまうことへの改善と効率的な時間管理能力の口上を図る。


本論文の構成は以下の通りである.
まず,\ref{cha:related} 章では,関連研究について述べる。
\ref{cha:Development} 章では,開発環境について述べる。
\ref{cha:function}章では,アプリケーションの機能について述べる.
\ref{cha:motion}章では、例を用いた実際の動作状況について述べる。
最後に,\ref{cha:conclusion} 章では,本論文のまとめと今後の課題を述べる.
\chapter{関連研究}
\label{cha:related}
人々は普段複数のタスクを抱えて生活している。複数のタスクに従事する現象はマルチタスキングと呼ばれていて、本来実行するタスクが他のタスクに妨害されることで生産性を低下させパフォーマンスを低下させる。
これら複数のタスクを管理するのに適したものがTODOリストである。TODOリストを使うことによって複数のタスクに順序性をもたせパフォーマンスの向上、タスクの継続の手助けしている。

しかし、TODOリストにはいくつかの欠点があり、タスクを管理することで逆に生産性が下がってしまうリスクが生じてしまう場合がある。タスクをリストに加えると実際には少し早めにやらなければならないタスクに関して余裕があるという感覚になりタスクをこなすスピードが低迷してしまう恐れがある。
タスクを可視化した場合簡単で早く終わるタスクばかり目についてしまいそればかりを終わらせてしまい、難しいタスクが長時間残ってしまう。
これらを改善するためにタスクの継続と利用頻度を向上させる工夫が必要になってくる。現在、TODOリストを長期間使用させるために様々なTODOリストアプリが作られている。

\cite{todometer}のtodometerというフリーのTODOリストではメータ表示でタスクの達成率が一目で分かり安くなっており、モチベーションを向上させることを目的としている。
リストの中から今日中にやるタスクを選択することができ、メータに黄色のラインが伸びる。今日中にやるべきタスクを完了させると黄色いラインの上に緑色のラインが増え、緑のラインの伸び具合によりその日に自分がどれだけタスクを完了したかを確認することができる。
残りのタスクがどれくらいあるかも確認できるのでやる気の継続にもつながる。

タブ型TODOリスト\cite{tab}という携帯アプリではタブ毎にタスクを分類することができ、仕事関連や家でやるべきことなどを分けることで情報をきれいに整理することができる。
タスクの完了ボタンを押すとタスク名が灰色になり色が変わるのでタスクが完了したかしてないかが一目で分かる。また、このアプリの大きな特徴は完了してないタスクの数がバッジとしてアプリアイコンの右上に出ることである。
携帯を開いた時に今日のタスクがいくつあるかを瞬時に確認することができるため、すぐタスクに取り掛かる気持ちになれる。

Wacca\cite{Wacca}というアプリは1日24時間を円形で表示し、その円形の中に円グラフ形式でタスクが表示される。タスクの始まりと終わりの部分に時間が表示されるのでそのタスクにどれくらいの時間が割かれているかが一目で確認できるのがこのアプリの最大の特徴である。
その日にやるべきことをすべてこの円グラフに入れるとどれだけ自由な時間があるか、どのくらい無駄な時間があるかなどが分かりより効率的な時間の使い方を身につけることができる。
タスクを色分けできるので種類を分けることもできる。

\newpage
\cite{report} では、効率的な時間管理術を養わせるためとTODOリストの使用を推奨させるために個人向けのTODOリスト利用促進システムを開発している。
この研究では自分のTODOリストに設定したタスクを他のユーザが見れるように公開サーバーに公開することができる。
公開したタスクは他ユーザからフィードバックを送信することができ、それによってそのタスクの未経験者も参考にすることができモチベーション維持を行っている。
またそのほかにも達成ポイントというのが備わっておりタスクを達成するとポイントが付与され設定した日時より後にタスクを達成するともらえるポイントが減少する、
タスクを公開することでポイントをもらえるなど継続して使用、日時を守ってタスクを終わらせられるようにするなどTODOリストの使用を維持させるための工夫がされている。
簡単なシステムの流れを図\ref{fig:example}に示す。
\newpage
\insertfigpng{example}{システムの流れ}
\chapter{XXX}
\label{cha:xxx}
このファイルは、ファイル名や章タイトル、そして、label を適宜書き換えること。

\insertfigeps{sample-eps}{eps 画像の貼り付けの例}
\insertfigpng{sample-png}{png 画像の貼り付けの例}
\input{tex/development}
\chapter{システム概要}
\label{cha:function}

\section{機能}
本アプリケーションで実装した機能は以下のとおりである。

\begin{itemize}
    \item データの入力
    \item データの編集
    \item 並び替え
    \item タイマー
\end{itemize}

\section{各機能の説明}
上で記述した機能を一つずつ解説していく。
\subsection{データの入力}

\subsection{データの編集}
\subsection{並び替え}
\subsection{タイマー}
\chapter{アプリケーションの動作}
\label{cha:motion}

実際に動かしているところを図を用いて説明。

\insertfigpng{add}{初期追加画面}
\insertfigpng{add2}{入力した後の追加画面}
\insertfigpng{home-add}{タスク追加後のホーム画面}
\insertfigpng{edit2}{初期編集画面}
\insertfigpng{edit3}{編集内容を入力した編集画面}
\insertfigpng{home-edit}{タスク編集後のホーム画面}
\chapter{まとめと今後の課題}
\label{cha:conclusion}

 本稿では,TODOリスト利用者のTODOリストの使用継続の維持と使用頻度の向上のため優先度機能の設定とタイマー機能が付いたTODOリストを開発した。
 まず、一般的なTODOリスト同様にタスクの追加と編集システムを追加した。
 そこに予想終了時間の設定と優先度を設定できるようにした。
 次にタイマー機能を追加した。
 
 今後の課題としては、今回自分の技術が拙くごく一般的なTODOリストに見劣りするものしか作れなかったことと本当に必要最低限の機能しか付け加えれなかったことである。
もう少し、リストの情報の整理に努めて作るならばリストで表示するだけでなくいくつかのタスクを選択して、それをフローチャートのようにタスクをつなげてその日にやることの順番を決めれるようにしたり、
カレンダーを表示してその日のタスクを表示するなど情報の可視化を手助けする視覚面でのシステムも追加するべきだった。今後より便利な機能を実装したいと考えている。

\acknowledgment
本研究と本論文を終えるにあたり、御指導、御教授を頂いた久松潤之准教授に
深く感謝致します。また、学生生活を通じて、基礎的な学問、学問に取り組む
姿勢を御教授頂いた、登尾啓史教授、升谷保博教
授、渡邊郁教授、南角茂樹教授、鴻巣敏之教授、北嶋暁教授、大西克彦
教授に深く感謝致します。

本研究期間中、本研究に対する貴重な御意見、御協力を頂きました久松研究室
の皆様に心から御礼申し上げます。


\bibliographystyle{junsrt}
\bibliography{bib/myrefs}


\end{document}
