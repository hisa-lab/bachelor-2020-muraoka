\chapter{TODOリストの開発}
\label{cha:Development}
開発したものの簡単な説明。

\section{開発環境}
本節では今回開発する際に使用した開発環境についての説明を記述する。
\subsection{python}
pythonとは1991年にグイド・ヴァンロッサムというプログラマーによって開発されオープンソースで運営されているプログラム言語である。
主な特徴としては、少ないコードで簡潔にプログラムをかけることと専門的なライブラリが豊富であることである。
プログラムが書きやすいことの一つに開発に役立つプログラムをまとめたものであるライブラリの数が数万にも及ぶため有効に活用することで
作りたいプログラムを比較的容易に作成することができる。

今回の開発には様々な機能を使うのでライブラリが多いpythonを選んだ。
\subsection{Django}
Djangoとはコンテンツ管理システムやソーシャルネットワーク、ニュースサイトなど質の高いWebアプリケーションを簡単に
短いコードで作成することができるpythonで実装されたフレームワークである。
Djangoはフルスタックのフレームワークであり多数の便利な機能を装備している。
Webアプリでよく使われる「ユーザ認証」「管理画面」などの機能はすでにあらかじめ備わっている。
さらにDjangoはモジュールの独立性が高く、メンテナンスや拡張が容易にできるようになっている。
一つのファイルに何百行も書くようなプログラムでもそれらを複数のファイルに細かく分けて記述することができるので、エラーを発見しやすくまた修正もしやすいので開発者にとても親切な設計となっている。

今回はwebアプリケーションを開発するのでこちらのフレームワークを選んだ。
\subsection{Jquery}
JqueryとはJavaScriptでできることをより簡単な記述で実現できるように設計されたJavaScriptライブラリである。
Jqueryを使うことでHTMLやCSSでのデザインを容易に変更することができ、本来のJavaScriptで記述すると膨大なコードになってしまうという容量の問題を解決することができる。
またJqueryをインストールせずともプログラム内にurlを記述することで使用することができるのでとても簡単である。

今回はデザインを変更したいのとJavaScriptで使いたいライブラリがあったのでこちらを使用した。
\newpage
\section{使用したモジュール}
それぞれの使用理由を記述。
\subsection{bootstrap}

\subsection{detepicker}

\subsection{message}

\subsection{models}

\subsection{urls}