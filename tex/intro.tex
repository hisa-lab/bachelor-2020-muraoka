\chapter{はじめに}
\label{cha:intro}

従来のTODOリストを使う目的はやるべきこと(タスク)をリストアップし、情報を整理することで時間を使い方を改善する。
複数の作業が重なったとき、何から進めればいいか混乱してしまい非効率になる可能性を防ぐことができる。
また、タスクを一つ一つこなしリストから消していくことでじぶんはこれだけやったんだと満足感が得られモチベーションの向上にもつながる。
情報を可視化し明確にすることで生産性を高めることを目的としている。さらに、自分のやるべきことと期限をメモ帳やカレンダーなどに書き込みそれを実行する習慣をつけることで
効率的な時間管理術を身につけることができる。

しかし、実際には時間のかかるタスクを残してしまいすぐ終わるものばかりをこなしてしまったり、リストに書くだけ書いてやらなくなる人が多い。
やるべきことをリストアップしてるだけでどれからやればいいかの優先順位がないので効率が悪くなります。どのくらいの時間で終わるかがわからないので優先順位をつける時も混乱してしまう。
先ほども言ったように時間のかかるような難しいタスクをのしがちになるので取り組んだが途中で投げ出してしまうというケースも少なくはない。

そこで、本研究では優先順位がなく非効率になることを改善する為、優先順位機能の開発をする。これによりどれから取り組めばいいかを可視化させ効率化を図る。

次に予想終了時間を設定できるようにする。優先順位機能を開発して優先順位をつけるといっても具体的な基準、目途がないとただ優先順位をつけるだけではタスクをこなすモチベーションにはつながらない。
予想終了時間を設定できるようにすることでよりはっきりと明瞭にし、情報の整理の手助けをすることでモチベーションの口上を図る。

次にタイマー機能の開発をする。取り組む時間を設定して、集中力をあげることを目的とする。時間を決めずにだらだらとタスクを行っても良いことはない。
取り組む時間を決めることで「その時間まで頑張ろう」という気持ちになり、無駄な時間を使わず少ない時間でタスクを終わらせられることを目標とする。
また、時間設定を繰り返し行うことでそれらにかかる時間をある程度予測できるようになり、より精密な時間管理を行うことができる。

これら3つの機能を追加することで、タスクをリストに書くだけでやらなくなってしまうことへの改善と効率的な時間管理能力の口上を図る。


本論文の構成は以下の通りである.
まず,\ref{cha:related} 章では,関連研究について述べる。
\ref{cha:Development} 章では,開発環境について述べる。
\ref{cha:function}章では,アプリケーションの機能について述べる.
\ref{cha:motion}章では、例を用いた実際の動作状況について述べる。
最後に,\ref{cha:conclusion} 章では,本論文のまとめと今後の課題を述べる.