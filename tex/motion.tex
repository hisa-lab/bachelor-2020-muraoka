\chapter{アプリケーションの動作}
\label{cha:motion}

実際に動かしているところを図を用いて説明。

ホームページ右上の+マークを押すとタスクを追加するための入力フォーム画面に遷移。各フォーム欄には初期値またはその欄に入力するものが記載されている。
実際の初期追加画面を図\ref{fig:add}に示す。
\insertfigpng{add}{初期追加画面}
\insertfigpng{add2}{入力した後の追加画面}
\insertfigpng{date}{カレンダー形式}


図\ref{fig:add2}のフォーム欄にタスク名、期限、予想終了時間と優先度を入力する。タスク名の欄には履歴機能がついており過去に入力したタスク名が欄下に出現し選択することができる。
期限と予想終了時間の欄は入力しやすいようにカレンダーから日付を入力できたりマウスで時間を入力できるようになっている。図を図\ref{fig:date}に示す。


\insertfigpng{home-add}{タスク追加後のホーム画面}
\insertfigpng{edit2}{初期編集画面}

タスクの設定を入力して図\ref{fig:add2}に表示されているAdd To LIstボタンを押すとデータが送信されホーム画面のリストの中にタスクが表示される。
タスクが追加された後のホーム画面を図\ref{fig:home-add}に示す。
リストに表示されているタスクの名前の部分がリンクになっておりクリックすることで編集画面に遷移する。
入力欄の初期値には選択したタスクのデータがすでに格納されている。
初期の編集画面を図\ref{fig:edit2}に示す。

\insertfigpng{edit3}{編集内容を入力した編集画面}
\insertfigpng{home-edit}{タスク編集後のホーム画面}

図\ref{fig:edit3}のように欄の中身を自由に変更することができる。変更した後Edit Itemボタンを押すことでホーム画面に遷移し変更したタスクが表示される。
タスクの設定の変更に成功しホームページに遷移した場合、「Item Has Been Edited!」というメッセージが画面上部に表示される。
タスク編集後のホーム画面を図\ref{fig:home-edit}に示す。

\insertfigpng{timer}{タイマー画面}

ホームページ上部にあるTimerという文字をクリックすると図\ref{fig:timer}に示しているタイマー画面に遷移することができる。
時間、分、秒の入力欄があり何も入力していないところには初期値0が格納されている。各欄の数値を決め、startボタンを押すことでカウントが始まる。
stopボタンでカウントの停止、restボタンでカウントを0に、各欄の数字の表示が0に変更される。
\insertfigpng{timer2}{タイマーの動作状況1}
\insertfigpng{timer3}{タイマーの動作状況2}
図\ref{fig:timer}のタイマーをカウントし続け秒の部分の値が0よりマイナスになった場合、カウントが停止する。時間または分の欄に0以上の数値が入っている場合はそちらの数値を1つ減らし、秒の数値が59になりカウントが継続される。
カウントされている状況を図\ref{fig:timer2},図\ref{fig:timer3}に示す。
