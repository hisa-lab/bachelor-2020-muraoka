\chapter{関連研究}
\label{cha:related}
人々は普段複数のタスクを抱えて生活している。複数のタスクに従事する現象はマルチタスキングと呼ばれていて、本来実行するタスクが他のタスクに妨害されることで生産性を低下させパフォーマンスを低下させる。
これら複数のタスクを管理するのに適したものがTODOリストである。TODOリストを使うことによって複数のタスクに順序性をもたせパフォーマンスの向上、タスクの継続の手助けしている。

しかし、TODOリストにはいくつかの欠点があり、タスクを管理することで逆に生産性が下がってしまうリスクが生じてしまう場合がある。タスクをリストに加えると実際には少し早めにやらなければならないタスクに関して余裕があるという感覚になりタスクをこなすスピードが低迷してしまう恐れがある。
タスクを可視化した場合簡単で早く終わるタスクばかり目についてしまいそればかりを終わらせてしまい、難しいタスクが長時間残ってしまう。
これらを改善するためにタスクの継続と利用頻度を向上させる工夫が必要になってくる。現在、TODOリストを長期間使用させるために様々なTODOリストアプリが作られている。

\cite{todometer}のtodometerというフリーのTODOリストではメータ表示でタスクの達成率が一目で分かり安くなっており、モチベーションを向上させることを目的としている。
リストの中から今日中にやるタスクを選択することができ、メータに黄色のラインが伸びる。今日中にやるべきタスクを完了させると黄色いラインの上に緑色のラインが増え、緑のラインの伸び具合によりその日に自分がどれだけタスクを完了したかを確認することができる。
残りのタスクがどれくらいあるかも確認できるのでやる気の継続にもつながる。

タブ型TODOリスト\cite{tab}という携帯アプリではタブ毎にタスクを分類することができ、仕事関連や家でやるべきことなどを分けることで情報をきれいに整理することができる。
タスクの完了ボタンを押すとタスク名が灰色になり色が変わるのでタスクが完了したかしてないかが一目で分かる。また、このアプリの大きな特徴は完了してないタスクの数がバッジとしてアプリアイコンの右上に出ることである。
携帯を開いた時に今日のタスクがいくつあるかを瞬時に確認することができるため、すぐタスクに取り掛かる気持ちになれる。

Wacca\cite{Wacca}というアプリは1日24時間を円形で表示し、その円形の中に円グラフ形式でタスクが表示される。タスクの始まりと終わりの部分に時間が表示されるのでそのタスクにどれくらいの時間が割かれているかが一目で確認できるのがこのアプリの最大の特徴である。
その日にやるべきことをすべてこの円グラフに入れるとどれだけ自由な時間があるか、どのくらい無駄な時間があるかなどが分かりより効率的な時間の使い方を身につけることができる。
タスクを色分けできるので種類を分けることもできる。

\cite{report} では、効率的な時間管理術を養わせるためとTODOリストの使用を推奨させるために個人向けのTODOリスト利用促進システムを開発している。
この研究では自分のTODOリストに設定したタスクを他のユーザが見れるように公開サーバーに公開することができる。
公開したタスクは他ユーザからフィードバックを送信することができ、それによってそのタスクの未経験者も参考にすることができモチベーション維持を行っている。
またそのほかにも達成ポイントというのが備わっておりタスクを達成するとポイントが付与され設定した日時より後にタスクを達成するともらえるポイントが減少する、
タスクを公開することでポイントをもらえるなど継続して使用、日時を守ってタスクを終わらせられるようにするなどTODOリストの使用を維持させるための工夫がされている。
簡単なシステムの流れを図\ref{fig:example}に示す。
%\insertfigpng{example}{システムの流れ}
