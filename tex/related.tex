\chapter{関連研究}
\label{cha:related}
人々は普段複数のタスクを抱えて生活している。複数のタスクに従事する現象はマルチタスキングと呼ばれていて、本来実行するタスクが他のタスクに妨害されることで生産性を低下させパフォーマンスを低下させる。
これら複数のタスクを管理するのに適したものがTODOリストである。TODOリストを使うことによって複数のタスクに順序性をもたせパフォーマンスの向上、タスクの継続の手助けしている。

しかし、TODOリストにはいくつかの欠点があり、タスクを管理することで逆に生産性が下がってしまうリスクが生じてしまう場合がある。タスクをリストに加えると実際には少し早めにやらなければならないタスクに関して余裕があるという感覚になりタスクをこなすスピードが低迷してしまう恐れがある。
タスクを可視化した場合簡単で早く終わるタスクばかり目についてしまいそればかりを終わらせてしまい、難しいタスクが長時間残ってしまう。
これらを改善するためにタスクの継続と利用頻度を向上させる工夫が必要になってくる。

例えば、\cite{report} では、効率的な時間管理術を養わせるためとTODOリストの使用を推奨させるために個人向けのTODOリスト利用促進システムを開発している。
この研究では自分のTODOリストに設定したタスクを他のユーザが見れるように公開サーバーに公開することができる。
公開したタスクは他ユーザからフィードバックを送信することができ、それによってそのタスクの未経験者も参考にすることができモチベーション維持を行っている。
またそのほかにも達成ポイントというのが備わっておりタスクを達成するとポイントが付与され設定した日時より後にタスクを達成するともらえるポイントが減少する、
タスクを公開することでポイントをもらえるなど継続して使用、日時を守ってタスクを終わらせられるようにするなどTODOリストの使用を維持させるための工夫がされている。

