\chapter{アプリケーションの設計}
\label{cha:function}

\section{システム概要}
本アプリケーションは優先順位の設定とタイマー機能が備わったTODOリストである。ホーム画面には自分で追加したタスクのリストが表示される。
ホーム画面上部はメニューバーとなっておりホームページのリンクとタイマー画面のリンクとタスク追加フォームへのボタンが備え付けられている。
タスク追加フォームで入力したデータをホームページのリストに表示させ編集フォームでそのデータの内容を変更して保存し、そのデータをまたホームページのリストに表示させる。
タイマー画面では時間、分、秒の設定ができ秒単位でカウントしていく。カウントが0になると通知が来るようになっている。

本アプリケーションで実装した機能は以下のとおりである。

\begin{itemize}
    \item データの編集
    \item 並び替え
    \item タイマー
\end{itemize}

\section{システム詳細}
\subsection{データの編集}
ホームページに表示されているリストの中にあるタスクのタスク名がリンクになっており、そのリンクをクリックすると編集ページに遷移する。
編集ページにはタスク名、期限、予想終了時間と優先度の入力フォームがあり、それぞれのフォーム欄にはホームページに表示されていたタスクのデータが格納される。
編集フォームを図\ref{fig:edit}に示す。\insertfigpng{edit}{編集フォームの図}

まず、タスク名のリンクを押すと変数が作成されクエリセットからインスタンスを取得するgetメソッドでその中に編集前のタスクの情報が格納され、render関数で指定したページに遷移する。
編集ページのそれぞれのフォームの欄には先ほどgetメソッドで取得した値が入力されている。新しいデータを入力した後、送信ボタンを押すと変数にデータが保存されホームページのリストに返される。
その時これらの処理が成功した場合、ホームページに成功のメッセージが表示されるようになっている。

\subsection{並び替え}
リスト上部に並び替えのドロップダウンボタンがありドロップダウンリストには複数のオプションがあり、ユーザーはその中から一つを選択することができる。
リストの中の一つを選択するとそこに示されているアクションが開始される。今回は日付の早い順遅い順、優先順位の高い順低い順の4つの項目を設定している。
項目を選択してアクションが起動した場合、urlsで紐づけたviewsに記述した関数が処理される。
データの編集の時と同じように変数が作成されデータが格納される。そのあとorderby関数を使用して昇順降順に並び替えることができる。

\subsection{タイマー}
時間、分、秒の数字を入力する欄があり、初期値は0となっている。こちらのタイマーはごく普通の一般的な仕組みのタイマーである。
startボタンを押すと1秒ずつカウントダウンされていきstopボタンを押すとカウントが停止し表示されている数字が変わらなくなる。
start,stopのどちらかが押されている場合にresetボタンを押すと変数の値が0になり、表示されている数字も0に戻るようになっている。



